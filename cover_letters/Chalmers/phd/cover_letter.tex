\title{Personal Letter for Chalmers PhD application}
\author{Attila Nagy}
\date{\today}

\documentclass[12pt]{article}

\begin{document}
\maketitle

\section*{Introduction and Qualifications}
My journey in the world of machine learning and artificial intelligence started during my Computer Science Master's program at the Gothenburg University when I decided to take the Artificial Intelligence course. In this course, I had the chance to be part of a team of 3 who developed a solution for a sentiment analyse of movie reviews and another solution for SHRDLite planner, a simpler version of SHRDLU\cite{win1970shrdlu}, using the GF\cite{gf} grammar interpreter for natural language processing. One of the supervisors for this course was Devdatt Dubhashi.
After my graduation, I directly started to work on Volvo's self-driving car project as a Software-Developer in the Sensor Fusion team. The work mainly involved post-processing of data from various sensors, such as HERE HD maps, Velodyne LiDAR, and development on Nvidia's Drive PX platform. As for the algorithm part, my involvement was less significant, but still relevant enough to understand and follow the development of Particle-Filtering for mapping and localization.
After my 2 years at Volvo Cars, I felt that I need change in my life, and therefore decided to try out the digital nomad life-style and travel in South America and Southeast Asia. During this period of my life, I finished Udacity's Deep Learning Nanodegree program, and gained practical knowledge by free-lancing on Codementor.
Currently, I work for a startup, RMBLStrip, located at Stena Center, next to Chalmer's Johanneberg campus. The main product of this company is an adaptive roof-deflector for the heavy truck industry. My role involves several parts of the development process, including data-collection and SW engineering, but the most interesting part is algorithm development. One of our solutions uses a classical Bayesian probabilistic method to find the optimal position for the roof-deflector.

\section*{Motivation and Career Goals}
\label{sec:motivation}
As for my motivation for becoming a PhD, in the recent years, starting with the Nanodegree on Udacity, my interests drifted more and more towards understanding the theoretics of machine learning. During these 4 months of study I had the opportunity to try and see many of the state-of-the-art algorithms and tools in action. One of these tools was Google's TensorFlow for implementing 1) a Convolutional Neural Network for image classification using the MNIST dataset, 2) script generation for a fictional conversation between Homer and Moe in Moe's tavern for The Simpsons TV series using a Recurrent Neural Network with word2vec embeddings, 3) language-translation from English to French using RNNs as encoder and decoder in an auto-encoder architecture, and finally 4) image-generation using Generative Adversarial Networks introduced by Ian Goodfellow himself as a guest lecturer.
Another essential tool was a GPU cluster necessary for training such algorithms. During these 4 assignments, I was first using Amazons Web Server, but due to unpredictable speed variations and connection breaks, I decided to use FloydHub instead. Their service proved to be much more reliable and fit better to my needs. However, I finally ended up using my own NVIDIA graphic card in my desktop PC due to its limitless accessibility and similar performance in terms of GPU speed.
All of the above mentioned projects were more focusing on practicalities than understanding the theoretical background of the algorithms. Most of the blogs that explained the essence of these algorithms barely scratched the surface of the topic and did not give a profound understanding in it. To gain such a knowledge, it is inevitable to dive deep into mathematics and research papers. For instance, all the above mentioned assignments involved the use of dropout to gain better regularization, but the course introduced barely the basic idea and neglected to go further in the details. My main motivation for seeking a PhD position is to have the chance to gain a deeper understanding in the underlying concepts of machine learning and to contribute to the next advancements in the field. In my current employment, these drives are let not to flourish due to incompatibility with the corporate interest, compelling me to pursue my interest in machine learning further as a hobby in my free time, after work. My motivation for applying for a PhD position at Chalmers in the field of mathematics in machine learning is to bring this hobby to the next level and have it as my main occupation.

\section*{Phd Positions}
From the 6 open PhD positions in machine learning at Chalmers the following two drew my attention:
\begin{enumerate}
    \item \textbf{Ref 20180233}: Understanding deep learning: From theory to algorithms
    \item \textbf{Ref 20180221}: Learning with noisy labels
\end{enumerate}

\subsection*{Understanding deep learning}
Understanding deep learning from a mathematical perspective is one of my key interesting in a PhD, and therefore this position is the most appealing to me. Furthermore, this is the position that is broad and flexible enough that I can absolutely see myself working within the next 5 years. Understanding the theoretical background in mathematics for a stochastic regularization method, such as dropout, would be an interesting approach to start with. Therefore, I feel the most inclined to work in this field from the currently available 6 open PhD positions on Chalmers.

\subsection*{Learning with noisy labels}
Learning with noisy labels was the other topic that immediately drew my attention. This is a recurrent topic in machine learning, significantly affecting all the segments of the field including both academia and industry. Furthermore, in this position I would have the chance to further polish my knowledge in probabilistic modelling that I recently gained at RMBLStrip working with a Bayesian Linear Regression algorithm. However, I do not feel fully qualified for this position since I have never officially taken any course in Information Theory. All my knowledge in this field is learned on-demand during work or my Udacity Nanodegree program. My favourite source for learning probabilistic programming during this period was the Bayesian Methods for Hackers book\cite{probabilistic_programming}, that is available as a Jupyter notebook online as well, making the whole learning process more interactive. The focus in this book is to provide a practical knowledge as well as the theory by having a handful of hands-on exercises using the PyMC3 Python library.

\subsection*{Other positions}
The other three positions in Ref 20180221 require deeper understanding in certain fields that I am not familiar with, such as particle physics and quantum theory. Therefore, I consider myself not qualified enough to apply for those positions.

\section*{References to contact}
Two persons are named in the CV that is attached to this application.

\subsection*{Milad Pouyanmehr}
Milad is my current team leader and one of the main share-holders at RMBLStrip. I work with him on a daily basis and therefore consider him as the best person to evaluate my performance.

\subsection*{Joakim Lin-S\"orstedt}
Joakim was the team leader of the Mapping and Localization team, the team that I was part of at Volvo Cars during my 2 years of employment. Currently, he is having a similar role at Zenuity working with localization and road estimation. Therefore, I consider him as the most competent person to evaluate my contribution to the team's achievements during those two years.

\bibliographystyle{abbrv}
\bibliography{references}

\end{document}

